% !TEX root = ./main.reuse.tex
%%%%%%%%%%%%%%%%%%%%%%%%%%%%%%%%%%%%%%%%%%%%%%%%%%%%%%%%%%%%%%%%%%%%%%%%%%%%%%%%
% An empty LaTeX project with useful packages and commands
%
% First created on: Nov 9, 2019
% Last updated on:  Feb 4, 2025
%%%%%%%%%%%%%%%%%%%%%%%%%%%%%%%%%%%%%%%%%%%%%%%%%%%%%%%%%%%%%%%%%%%%%%%%%%%%%%%%
\def\mode{lncs}
%%%%%%%%%%%%%%%%%%%%%%%%%%%%%%%%%%%%%%%%%%%%%%%%%%%%%%%%%%%%%%%%%%%%%%%%%%%%%%%%
% Style
%
% First created on: Jan 5, 2023
% Last updated on:  Dec 30, 2023
%%%%%%%%%%%%%%%%%%%%%%%%%%%%%%%%%%%%%%%%%%%%%%%%%%%%%%%%%%%%%%%%%%%%%%%%%%%%%%%%

%%% CONFIGURE DOCUMENTCLASS%%%%%%%%%%%%%%%%%%%%%%%%%%%%%%%%%%%%%%%%%%%%%%%%%%%%%
\def\icpm{%
\documentclass[conference]{./tpl/icpm/IEEEtran} % ICPM
\bibliographystyle{./tpl/icpm/IEEEtran}
\usepackage[noadjust]{cite}	% fix citations form [1],[2],[3] to [1-3].
\renewcommand{\citepunct}{,\penalty\citepunctpenalty\,}
\renewcommand{\citedash}{--}% optionally
\usepackage[firstpage]{draftwatermark}
\SetWatermarkAngle{0}
\SetWatermarkFontSize{12pt}
\SetWatermarkHorCenter{170.5mm}
\SetWatermarkVerCenter{264mm}
\SetWatermarkLightness{0.6}
\SetWatermarkText{Draft{,} \monthyeardate\today}
\usepackage[pdfsubject={},
            pdftitle={},
            pdfauthor={},
            pdfkeywords={},
            colorlinks=false,
            pdfborder={0.25 0.25 0.25}]{hyperref}
\hypersetup{bookmarksdepth}
}
\def\lncs{% lnbip
\documentclass[runningheads,orivec]{./tpl/lncs/llncs}		% LNCS, LNBIP
\setlength{\paperheight}{232.8mm}
\setlength{\paperwidth}{151.5mm}
\setlength\voffset     {-21mm}
\setlength\hoffset     {-34mm}
\bibliographystyle{./tpl/lncs/splncs04nat}
\usepackage[firstpage]{draftwatermark}
\SetWatermarkAngle{0}
\SetWatermarkFontSize{10pt}
\SetWatermarkHorCenter{153mm}
\SetWatermarkVerCenter{240mm}
\SetWatermarkLightness{0.6}
\SetWatermarkText{Draft{,} \monthyeardate\today}
\usepackage[pdfsubject={},
            pdftitle={},
            pdfauthor={},
            pdfkeywords={},
            colorlinks=false,
            pdfborder={0.25 0.25 0.25}]{hyperref}
\hypersetup{bookmarksdepth}
}
\def\article{%
\documentclass{article}	 % article
\setlength{\paperheight} {232.8mm}
\setlength{\paperwidth}  {151.5mm}
\setlength\voffset       {-26mm}
\setlength\hoffset       {-32mm}
\bibliographystyle{plain}
\usepackage[firstpage]{draftwatermark}
\SetWatermarkAngle{0}
\SetWatermarkFontSize{10pt}
\SetWatermarkHorCenter{153mm}
\SetWatermarkVerCenter{247mm}
\SetWatermarkLightness{0.6}
\SetWatermarkText{Draft{,} {\monthyeardate\today}}
\usepackage[pdfsubject={},
            pdftitle={},
            pdfauthor={},
            pdfkeywords={},
            colorlinks=false,
            pdfborder={0.25 0.25 0.25}]{hyperref}
\hypersetup{bookmarksdepth}
}
\def\tkde{%
\documentclass[10pt,journal,compsoc]{./tpl/tkde/IEEEtran} % TKDE
\bibliographystyle{./tpl/tkde/IEEEtran}
\renewcommand{\citepunct}{,\penalty\citepunctpenalty\,}
\renewcommand{\citedash}{--}% optionally
\ifCLASSOPTIONcompsoc
  % The IEEE Computer Society needs nocompress option
  % requires cite.sty v4.0 or later (November 2003)
  \usepackage[nocompress]{cite}
\else
  % normal IEEE
  \usepackage{cite}
\fi
\usepackage[firstpage]{draftwatermark}
\SetWatermarkAngle{0}
\SetWatermarkFontSize{12pt}
\SetWatermarkHorCenter{182mm}
\SetWatermarkVerCenter{269mm}
\SetWatermarkLightness{0.6}
\SetWatermarkText{Draft{,} \monthyeardate\today}
\usepackage[pdfsubject={},
            pdftitle={},
            pdfauthor={},
            pdfkeywords={},
            colorlinks=false,
            pdfborder={0.25 0.25 0.25}]{hyperref}
\hypersetup{bookmarksdepth}
}
\def\aaai{%
\documentclass[letterpaper]{article} % DO NOT CHANGE THIS
%\bibliographystyle{./tpl/aaai/aaai23}
\usepackage[submission]{./tpl/aaai/aaai23}  % DO NOT CHANGE THIS
\usepackage{times}  % DO NOT CHANGE THIS
\usepackage{helvet}  % DO NOT CHANGE THIS
\usepackage{courier}  % DO NOT CHANGE THIS
\usepackage[hyphens]{url}  % DO NOT CHANGE THIS
\usepackage{graphicx} % DO NOT CHANGE THIS
\urlstyle{rm} % DO NOT CHANGE THIS
\def\UrlFont{\rm}  % DO NOT CHANGE THIS
\usepackage{natbib}  % DO NOT CHANGE THIS AND DO NOT ADD ANY OPTIONS TO IT
\usepackage{caption} % DO NOT CHANGE THIS AND DO NOT ADD ANY OPTIONS TO IT
\frenchspacing  % DO NOT CHANGE THIS
\setlength{\pdfpagewidth}{8.5in} % DO NOT CHANGE THIS
\setlength{\pdfpageheight}{11in} % DO NOT CHANGE THIS
%
% These are are recommended to typeset listings but not required. See the subsubsection on listing. Remove this block if you don't have listings in your paper.
%\usepackage{newfloat}
%\usepackage{listings}
%\DeclareCaptionStyle{ruled}{labelfont=normalfont,labelsep=colon,strut=off} % DO NOT CHANGE THIS
%\lstset{%
	%basicstyle={\footnotesize\ttfamily},% footnotesize acceptable for monospace
	%numbers=left,numberstyle=\footnotesize,xleftmargin=2em,% show line numbers, remove this entire line if you don't want the numbers.
	%aboveskip=0pt,belowskip=0pt,%
	%showstringspaces=false,tabsize=2,breaklines=true}
%\floatstyle{ruled}
%\newfloat{listing}{tb}{lst}{}
%\floatname{listing}{Listing}
%
% Keep the \pdfinfo as shown here. There's no need
% for you to add the /Title and /Author tags.
%\pdfinfo{
%/TemplateVersion (2023.1)
%}
% DISALLOWED PACKAGES
% \usepackage{authblk} -- This package is specifically forbidden
% \usepackage{balance} -- This package is specifically forbidden
% \usepackage{color (if used in text)
% \usepackage{CJK} -- This package is specifically forbidden
% \usepackage{float} -- This package is specifically forbidden
% \usepackage{flushend} -- This package is specifically forbidden
% \usepackage{fontenc} -- This package is specifically forbidden
% \usepackage{fullpage} -- This package is specifically forbidden
% \usepackage{geometry} -- This package is specifically forbidden
% \usepackage{grffile} -- This package is specifically forbidden
% \usepackage{hyperref} -- This package is specifically forbidden
% \usepackage{navigator} -- This package is specifically forbidden
% (or any other package that embeds links such as navigator or hyperref)
% \indentfirst} -- This package is specifically forbidden
% \layout} -- This package is specifically forbidden
% \multicol} -- This package is specifically forbidden
% \nameref} -- This package is specifically forbidden
% \usepackage{savetrees} -- This package is specifically forbidden
% \usepackage{setspace} -- This package is specifically forbidden
% \usepackage{stfloats} -- This package is specifically forbidden
% \usepackage{tabu} -- This package is specifically forbidden
% \usepackage{titlesec} -- This package is specifically forbidden
% \usepackage{tocbibind} -- This package is specifically forbidden
% \usepackage{ulem} -- This package is specifically forbidden
% \usepackage{wrapfig} -- This package is specifically forbidden
% DISALLOWED COMMANDS
% \nocopyright -- Your paper will not be published if you use this command
% \addtolength -- This command may not be used
% \balance -- This command may not be used
% \baselinestretch -- Your paper will not be published if you use this command
% \clearpage -- No page breaks of any kind may be used for the final version of your paper
% \columnsep -- This command may not be used
% \newpage -- No page breaks of any kind may be used for the final version of your paper
% \pagebreak -- No page breaks of any kind may be used for the final version of your paperr
% \pagestyle -- This command may not be used
% \tiny -- This is not an acceptable font size.
% \vspace{- -- No negative value may be used in proximity of a caption, figure, table, section, subsection, subsubsection, or reference
% \vskip{- -- No negative value may be used to alter spacing above or below a caption, figure, table, section, subsection, subsubsection, or reference
%\setcounter{secnumdepth}{0} %May be changed to 1 or 2 if section numbers are desired.
\usepackage[firstpage]{draftwatermark}
\SetWatermarkAngle{0}
\SetWatermarkFontSize{12pt}
\SetWatermarkHorCenter{182mm}
\SetWatermarkVerCenter{269mm}
\SetWatermarkLightness{0.6}
\SetWatermarkText{Draft{,} \monthyeardate\today}
}
\csname\mode\endcsname
%%%%%%%%%%%%%%%%%%%%%%%%%%%%%%%%%%%%%%%%%%%%%%%%%%%%%%%%%%%%%%%%%%%%%%%%%%%%%%%%
%%%%%%%%%%%%%%%%%%%%%%%%%%%%%%%%%%%%%%%%%%%%%%%%%%%%%%%%%%%%%%%%%%%%%%%%%%%%%%%%
% Configuration constants
%
% First created on: Dec 2, 2019
% Last updated on:  Jul 19, 2020
% Prepared by Artem Polyvyanyy
%
% Email:    artem.polyvyanyy@gmail.com 
% Homepage: http://polyvyanyy.com

%%%%%%%%%%%%%%%%%%%%%%%%%%%%%%%%%%%%%%%%%%%%%%%%%%%%%%%%%%%%%%%%%%%%%%%%%%%%%%%%

% Authors' ORCIDs
\newcommand{\orcidartem}		{\href{https://orcid.org/0000-0002-7672-1643}{\protect\includegraphics[scale=0.05]{fig/orcid}}}	% Artem Polyvyanyy
\newcommand{\orcidalistair}	{\href{https://orcid.org/0000-0002-6638-0232}{\protect\includegraphics[scale=0.05]{fig/orcid}}}	% Alistair Moffat
\newcommand{\orcidluciano}	{\href{https://orcid.org/0000-0001-9076-903X}{\protect\includegraphics[scale=0.05]{fig/orcid}}}	% Luciano Garcia-Banuelos
\newcommand{\orcidmatthias}	{\href{https://orcid.org/0000-0003-3325-7227}{\protect\includegraphics[scale=0.05]{fig/orcid}}}	% Matthias Weidlich
\newcommand{\orcidandreas}	{\href{https://orcid.org/0000-0002-0537-6598}{\protect\includegraphics[scale=0.05]{fig/orcid}}} % Andreas Solti
\newcommand{\orcidanna}			{\href{https://orcid.org/0000-0002-5088-7602}{\protect\includegraphics[scale=0.05]{fig/orcid}}} % Anna Kalenkova
\newcommand{\orcidjan}			{\href{https://orcid.org/0000-0002-7260-524X}{\protect\includegraphics[scale=0.05]{fig/orcid}}} % Jan Mendling
\newcommand{\orcidclaudio}	{\href{https://orcid.org/0000-0001-5570-0475}{\protect\includegraphics[scale=0.05]{fig/orcid}}} % Claudio Di Ciccio
\newcommand{\orcidsander}		{\href{https://orcid.org/0000-0002-5201-7125}{\protect\includegraphics[scale=0.05]{fig/orcid}}} % Sander Leemans

%%%%%%%%%%%%%%%%%%%%%%%%%%%%%%%%%%%%%%%%%%%%%%%%%%%%%%%%%%%%%%%%%%%%%%%%%%%%%%%%

% constants
\newcommand{\identifiers}	{{\ensuremath \mathcal{I}}}
\newcommand{\labels}			{{\ensuremath \Lambda}}
\newcommand{\actions}			{{\ensuremath \Lambda}}

% math constants
\newcommand{\natnum}					{\ensuremath \mathbb{N}}
\newcommand{\natnumwithzero}	{\ensuremath \mathbb{N}_0}
\newcommand{\nonnegrealnumbers}	{\ensuremath \mathbb{R}^+}

% latin abbreviations:
% https://en.wikipedia.org/wiki/List_of_Latin_abbreviations
\newcommand{\ie}					{i.e.,~}
\newcommand{\eg}					{e.g.,~}
\newcommand{\see}					{cf.~}

%%%%%%%%%%%%%%%%%%%%%%%%%%%%%%%%%%%%%%%%%%%%%%%%%%%%%%%%%%%%%%%%%%%%%%%%%%%%%%%%

%\newcommand{\orcidiconartem}	{\href{https://orcid.org/0000-0002-7672-1643}{\textcolor{orcidlogocol}{\aiOrcid}}}

%%%%%%%%%%%%%%%%%%%%%%%%%%%%%%%%%%%%%%%%%%%%%%%%%%%%%%%%%%%%%%%%%%%%%%%%%%%%%%%%
%%%%%%%%%%%%%%%%%%%%%%%%%%%%%%%%%%%%%%%%%%%%%%%%%%%%%%%%%%%%%%%%%%%%%%%%%%%%%%%%
% Configuration constants
%
% First created on: Nov 9, 2019
% Last updated on:  Jul 10, 2021
% Prepared by Artem Polyvyanyy
%
% Email:    artem.polyvyanyy@gmail.com 
% Homepage: http://polyvyanyy.com
%%%%%%%%%%%%%%%%%%%%%%%%%%%%%%%%%%%%%%%%%%%%%%%%%%%%%%%%%%%%%%%%%%%%%%%%%%%%%%%%

%%% CONFIGURE DOCUMENTCLASS%%%%%%%%%%%%%%%%%%%%%%%%%%%%%%%%%%%%%%%%%%%%%%%%%%%%%
\def\icpm{%
\documentclass[conference]{./tpl/icpm/IEEEtran}		% ICPM
\bibliographystyle{./tpl/icpm/IEEEtran}
\usepackage[noadjust]{cite}	% fix citations form [1],[2],[3] to [1-3].
\renewcommand{\citepunct}{,\penalty\citepunctpenalty\,}
\renewcommand{\citedash}{--}% optionally
\usepackage[firstpage]{draftwatermark}
\SetWatermarkAngle{0}
\SetWatermarkFontSize{12pt}
\SetWatermarkHorCenter{170.5mm}
\SetWatermarkVerCenter{264mm}
\SetWatermarkLightness{0.6}
\SetWatermarkText{Draft{,} \monthyeardate\today}
}
\def\lncs{% lnbip
\documentclass[runningheads,orivec]{./tpl/lncs/llncs}		% LNCS, LNBIP
\setlength{\paperheight}{232.8mm}
\setlength{\paperwidth}{151.5mm}
\setlength\voffset     {-23mm}
\setlength\hoffset     {-34mm}
\bibliographystyle{./tpl/lncs/splncs04}
\usepackage[firstpage]{draftwatermark}
\SetWatermarkAngle{0}
\SetWatermarkFontSize{10pt}
\SetWatermarkHorCenter{153mm}
\SetWatermarkVerCenter{240mm}
\SetWatermarkLightness{0.6}
\SetWatermarkText{Draft{,} \monthyeardate\today}
}
\def\article{%
\documentclass{article}	 % article
\setlength{\paperheight} {232.8mm}
\setlength{\paperwidth}  {151.5mm}
\setlength\voffset       {-26mm}
\setlength\hoffset       {-32mm}
\bibliographystyle{plain}
\usepackage[firstpage]{draftwatermark}
\SetWatermarkAngle{0}
\SetWatermarkFontSize{10pt}
\SetWatermarkHorCenter{153mm}
\SetWatermarkVerCenter{247mm}
\SetWatermarkLightness{0.6}
\SetWatermarkText{Draft{,} {\monthyeardate\today}}
}
\csname\mode\endcsname
%%%%%%%%%%%%%%%%%%%%%%%%%%%%%%%%%%%%%%%%%%%%%%%%%%%%%%%%%%%%%%%%%%%%%%%%%%%%%%%%

% Configure todo items
\newif\ifshowtodos
\showtodostrue		% show todo items
%\showtodosfalse	% do not show todo items

\newcommand{\authorartem}		 {Artem~Polyvyanyy}
\newcommand{\authoralistair} {Alistair~Moffat}
\newcommand{\authorluciano}	 {Luciano~Garc{\'{\i}}a{-}Ba{\~{n}}uelos}

% Use APA title case for the titles: https://capitalizemytitle.com/
\newcommand{\articletitle}   {An Article on an Important Topic}
\newcommand{\articlesubjet}  {Computer Science, Process Mining, Process Querying, Information Systems}
\newcommand{\articleauthors} {\authorartem}

% Set counters
% The depth of the bookmarks
\setcounter{tocdepth}{3}
\setcounter{footnote}{0}

%%%%%%%%%%%%%%%%%%%%%%%%%%%%%%%%%%%%%%%%%%%%%%%%%%%%%%%%%%%%%%%%%%%%%%%%%%%%%%%%
%%%%%%%%%%%%%%%%%%%%%%%%%%%%%%%%%%%%%%%%%%%%%%%%%%%%%%%%%%%%%%%%%%%%%%%%%%%%%%%%
% Improting useful packages and defining useful commands for LaTeX
%
% First created on: Dec 2, 2019
% Last updated on:  Dec 30, 2023
%%%%%%%%%%%%%%%%%%%%%%%%%%%%%%%%%%%%%%%%%%%%%%%%%%%%%%%%%%%%%%%%%%%%%%%%%%%%%%%%

%%%%%%%%%%%%%%%%%%%%%%%%%%%%%%%%%%%%%%%%%%%%%%%%%%%%%%%%%%%%%%%%%%%%%%%%%%%%%%%%
% USE PACKAGES
%%%%%%%%%%%%%%%%%%%%%%%%%%%%%%%%%%%%%%%%%%%%%%%%%%%%%%%%%%%%%%%%%%%%%%%%%%%%%%%%

% Use natbib package
\usepackage[square,sort&compress,comma,numbers]{natbib}
\renewcommand{\bibfont}{\small}
\renewcommand{\bibsep}{0pt}
\renewcommand{\bibsection}{\section*{References}\vspace{-2mm}}

%-------------------------------------------------------------------------------
%\usepackage{amsthm}
%\usepackage{amssymb}
%-------------------------------------------------------------------------------
% A comprehensive (SI) units package
% https://ctan.org/pkg/siunitx?lang=en
%\usepackage{siunitx}
%\sisetup{
%group-separator={,},
%round-mode=places,
%round-precision=3,
%table-format=1.3,
%}
%-------------------------------------------------------------------------------
% Useful table packages
% booktabs - the package enhances the quality of tables
% https://ctan.org/pkg/booktabs?lang=en
% multirow - create tabular cells spanning multiple rows
% https://ctan.org/pkg/multirow?lang=en
\usepackage{booktabs}
\usepackage{multirow}
%-------------------------------------------------------------------------------
% The package typesets fractions “nicely” — in the form ‘a/b’
% https://ctan.org/pkg/nicefrac?lang=en
% USAGE: $\nicefrac{1}{2}$
\usepackage{nicefrac}
%-------------------------------------------------------------------------------
% https://ctan.org/pkg/ifthen?lang=en
\usepackage{ifthen}
%-------------------------------------------------------------------------------
% https://en.wikibooks.org/wiki/LaTeX/Colors
\usepackage[usenames,dvipsnames]{xcolor}
%-------------------------------------------------------------------------------
%https://ctan.org/pkg/algorithm2e?lang=en
\usepackage[algoruled,vlined,linesnumbered]{algorithm2e}
%-------------------------------------------------------------------------------
% https://ctan.org/pkg/amsmath?lang=en
%
% Recommended as an adjunct to serious mathematical typesetting in LATEX.
\usepackage{amsmath}
%-------------------------------------------------------------------------------
% https://ctan.org/pkg/amsfonts?lang=en
%
% An extended set of fonts for use in mathematics, including: extra mathematical 
% symbols; blackboard bold letters (uppercase only); fraktur letters; subscript 
% sizes of bold math italic and bold Greek letters; subscript sizes of large 
% symbols such as sum and product; added sizes of the Computer Modern small caps
% font; cyrillic fonts (from the University of Washington); Euler mathematical 
% fonts.
\usepackage{amsfonts}
%-------------------------------------------------------------------------------
\usepackage{mdframed}
%-------------------------------------------------------------------------------
% https://ctan.org/pkg/paralist?lang=en
\usepackage{paralist}
\setdefaultleftmargin{1em}{1em}{1em}{1em}{1em}{1em}
%-------------------------------------------------------------------------------
\usepackage{datetime}
\newdateformat{monthyeardate}{\monthname[\THEMONTH] \THEYEAR}
%-------------------------------------------------------------------------------
\usepackage[algoruled]{algorithm2e}
%-------------------------------------------------------------------------------
% Manage to do items
% https://ctan.org/pkg/todonotes?lang=en
\ifshowtodos
\usepackage[textsize=scriptsize,colorinlistoftodos]{todonotes}
\else
\usepackage[textsize=scriptsize,colorinlistoftodos,disable]{todonotes} % disable all todo notes
\fi
% useful commands
% \missingfigure{My comment} - specify a missing figure
% \listoftodos - list all todos in the current document
\let\todonote\todo
\renewcommand{\todo}[2]{\todonote[inline,color=red!20]{TODO (#1): #2}}
\newcommand{\done}[2]{\todonote[inline,color=green!20]{DONE (#1): #2}}
\newcommand{\chat}[2]{\todonote[inline,color=black!20,nolist]{#1: #2}}
\newcommand{\idea}[2]{\todonote[inline,color=blue!20,nolist]{IDEA (#1): #2}}
%-------------------------------------------------------------------------------
\usepackage{tikz,pgf,xcolor}
\usetikzlibrary{arrows,automata,trees,plotmarks,shadows,shapes}
\usepackage{pgfplots}
\usetikzlibrary{pgfplots.groupplots}
%-------------------------------------------------------------------------------
% TODO: check the usage
%\usepackage{url}
%\urlstyle{same}
%-------------------------------------------------------------------------------
% TODO: check the usage
\usepackage{mathtools} 
%-------------------------------------------------------------------------------
% Manage references
% https://ctan.org/tex-archive/macros/latex/contrib/cleveref
\usepackage[capitalise,nameinlink]{cleveref}
%-------------------------------------------------------------------------------
% Flow text around small figures
\usepackage{wrapfig}
%-------------------------------------------------------------------------------

% Can be useful
%http://anorien.csc.warwick.ac.uk/mirrors/CTAN/macros/latex/contrib/constants/constants.pdf

%%%%%%%%%%%%%%%%%%%%%%%%%%%%%%%%%%%%%%%%%%%%%%%%%%%%%%%%%%%%%%%%%%%%%%%%%%%%%%%%
% CONSTANTS
%%%%%%%%%%%%%%%%%%%%%%%%%%%%%%%%%%%%%%%%%%%%%%%%%%%%%%%%%%%%%%%%%%%%%%%%%%%%%%%%

\definecolor{mybluecolor}{RGB}{50,106,218}
\definecolor{myredcolor}{RGB}{176,53,53}
\definecolor{mygreencolor}{RGB}{93,172,0}
\definecolor{myyellowcolor}{RGB}{255,163,34}
\definecolor{mypurplecolor}{RGB}{86,35,132}
\definecolor{mytealcolor}{RGB}{30,161,165}

\newcommand{\redtext}[1]{\textcolor{myredcolor}{{#1}}}
\newcommand{\bluetext}[1]{\textcolor{mybluecolor}{{#1}}}

%%%%%%%%%%%%%%%%%%%%%%%%%%%%%%%%%%%%%%%%%%%%%%%%%%%%%%%%%%%%%%%%%%%%%%%%%%%%%%%%
% DEFINE NEW COMMANDS
%%%%%%%%%%%%%%%%%%%%%%%%%%%%%%%%%%%%%%%%%%%%%%%%%%%%%%%%%%%%%%%%%%%%%%%%%%%%%%%%

\newcommand{\splitatcommas}[1]{%
  \begingroup
  \ifnum\mathcode`,="8000
  \else
    \begingroup\lccode`~=`, \lowercase{\endgroup
      \edef~{\mathchar\the\mathcode`, \penalty0 \noexpand\hspace{-1pt plus 3em}}%
    }\mathcode`,="8000
  \fi
  #1%
  \endgroup
}

%-------------------------------------------------------------------------------
% Arithmetic operations
\newcommand{\mult}{{\ensuremath \,\cdot\,}}

% Math symbols
\providecommand{\bigsum}{\ensuremath \mbox{\LARGE $\mathsurround0pt\sum$}}
\renewcommand	 {\bigsum}{\ensuremath \mbox{\LARGE $\mathsurround0pt\sum$}}

% Math functions
\newcommand{\func}[3]{{{#1}:{#2} \rightarrow {#3}}}
\newcommand{\funcCall}[2]{{\ensuremath {\mathit{#1}}_{\!}\left({#2}\right)}}
\newcommand{\funcDomain}[1]{{\ensuremath {\mathit{dom}}_{\!}\left({#1}\right)}}
\newcommand{\funcRange}[1]{{\ensuremath {\mathit{rng}}_{\!}\left({#1}\right)}}

% Math paired delimiters
\DeclarePairedDelimiter\ceil{\lceil}{\rceil}
\DeclarePairedDelimiter\floor{\lfloor}{\rfloor}

% Math intervals
\newcommand{\intintervalcc}[2]{{\ensuremath \left[#1 \,..\, #2\right]}}
\newcommand{\intervalcc}[2]{{\ensuremath \left[#1, #2\right]}}

% Math powerset
\providecommand{\powerset}[1]{{\ensuremath \mathcal{P}\!\left({#1}\right)}}
\renewcommand{\powerset}[1]{{\ensuremath \mathcal{P}\!\left({#1}\right)}}

% Math multiset
\providecommand{\support}[1]{{\ensuremath \mathit{Supp}\left({#1}\right)}}
\renewcommand{\support}[1]{{\ensuremath \mathit{Supp}\left({#1}\right)}}
\providecommand{\cardinality}[1]{\ensuremath \left|{#1}\right|}
\renewcommand{\cardinality}[1]{\ensuremath \left|{#1}\right|}
\providecommand{\multiplicity}[2]{\ensuremath \funcCall{m_{#1}}{#2}}
\renewcommand{\multiplicity}[2]{\ensuremath \funcCall{m_{#1}}{#2}}
 \newcommand{\mset}[1] {\ensuremath [\splitatcommas{#1}]}
\newcommand{\msetel}[2]{{\ensuremath {{#1}^{#2}}}}

% Math sets
\newcommand{\set}[1]{\ensuremath \left\{\splitatcommas{#1}\right\}}
\newcommand{\setbuilder}[2]{\ensuremath \left\{ #1 \,|\, #2 \right\}}

% Math tuple functions
\newcommand{\pair}[2]{\ensuremath \left(\splitatcommas{#1, #2}\right)}
\newcommand{\triple}[3]{\ensuremath \left(\splitatcommas{#1, #2, #3}\right)}
\newcommand{\tuple}[1]{\ensuremath \left(\splitatcommas{#1}\right)}

% Math sequences
\newcommand{\kleenestar}[1]{{\ensuremath {#1}^{*}}}
\newcommand{\emptysequence}{{\ensuremath \epsilon}}
\newcommand{\sequence}[1]{\ensuremath \langle\splitatcommas{#1}\rangle}
\newcommand{\seqLength}[1]{\ensuremath \left|{#1}\right|}
\newcommand{\concat}[2]{\ensuremath #1 \circ #2}
\newcommand{\subseq}[3]{\ensuremath \funcCall{subseq}{#1, #2, #3}}
\newcommand{\subseqid}[3]{\ensuremath \triple{#1}{#2}{#3}}
\newcommand{\crossover}[3]{\ensuremath #1 \,\otimes\, #2 \,=\, #3}
\newcommand{\crossoverop}[2]{\ensuremath #1 \otimes #2}

% Petri nets
\newcommand{\enabled}[3]{\ensuremath \pair{#1}{#2}\left[#3\right\rangle}
\newcommand{\occurrence}[4]{\ensuremath \pair{#1}{#2}\left[#3\right\rangle\pair{#1}{#4}}
\newcommand{\leadsfromto}[4]{\ensuremath \pair{#1}{#2}\left[#3\right\rangle\pair{#1}{#4}}

% Predicates
\newcommand{\predicate}[3]{\ensuremath #1 \, #2 : #3}

% Theorems and proofs
\providecommand{\qed}{\hfill\ensuremath{\filledsquare}}
\renewcommand  {\qed}{\hfill\ensuremath{\filledsquare}}
\providecommand{\holds}{\ensuremath :}
\renewcommand	 {\holds}{\ensuremath :}
\providecommand{\iff}{\ensuremath \Leftrightarrow}
\renewcommand	 {\iff}{\ensuremath \Leftrightarrow}
\providecommand{\implies}{\ensuremath \Rightarrow}
\renewcommand	 {\implies}{\ensuremath \Rightarrow}
\providecommand{\bigforall}{\ensuremath \mbox{\LARGE $\mathsurround0pt\forall$}}
\renewcommand	 {\bigforall}{\ensuremath \mbox{\LARGE $\mathsurround0pt\forall$}}

% Logs
\newcommand{\action}[1]{\texttt{#1}}
\newcommand{\trace}[1]{\texttt{#1}}

%-------------------------------------------------------------------------------

% Figure functions
% Use: \fig{pos}{vspace(mm)}{scale}{trim(0 0 0 0)}{figure}{vspace(mm)}{Caption}{vspace(mm)}{fig:label}
\newcommand{\fig}[9]{\begin{figure}[#1]
\vspace{#2mm}
\begin{center}
	\includegraphics[scale=#3,trim=#4]{#5}
\end{center}
\vspace{#6mm}
\caption{#7.}
\vspace{#8mm}
\label{#9}
\end{figure}}

%-------------------------------------------------------------------------------

% An initial attempt to fix orphan lines in a pargraph
\newcommand\fixit[2][.05]{%
  \setbox0=\hbox{\hspace{\parindent}#2}\fixithelp{#1}{#2}}
\newcommand\fixithelp[2]{%
  \wd0=\dimexpr\wd0-\linewidth\relax%
  \ifdim\wd0>0pt\relax%
    \fixithelp{#1}{#2}%
  \else%
    \wd0=\dimexpr\wd0+\linewidth\relax
    \ifdim\wd0>.9\linewidth\relax%
      {\parfillskip0pt\relax#2\par}%
    \else%
      \ifdim\wd0>.8\linewidth\relax%
        {\parfillskip0pt\relax#2\hspace{.2\linewidth}\par}%
      \else%
        \ifdim\wd0<#1\linewidth\relax%
          {\parfillskip0pt\relax#2\par}%
        \else%
          \ifdim\wd0<.2\linewidth\relax%
            {\parfillskip0pt\relax#2\hspace{.8\linewidth}\mbox{}\par}%
          \else%
            #2%
          \fi
        \fi
      \fi
    \fi
  \fi%
}

%%%%%%%%%%%%%%%%%%%%%%%%%%%%%%%%%%%%%%%%%%%%%%%%%%%%%%%%%%%%%%%%%%%%%%%%%%%%%%%%
% Theorem environments
%
% First created on: Dec 2, 2019
% Last updated on:  Dec 17, 2019
% Prepared by Artem Polyvyanyy
%
% Email:    artem.polyvyanyy@gmail.com 
% Homepage: http://polyvyanyy.com
%%%%%%%%%%%%%%%%%%%%%%%%%%%%%%%%%%%%%%%%%%%%%%%%%%%%%%%%%%%%%%%%%%%%%%%%%%%%%%%%

\newtheorem{mytheorem}		{Theorem}
\newtheorem{mydefinition}	{Definition}
\newtheorem{mylemma}			{Lemma}
\newtheorem{myproposition}{Proposition}
\newtheorem{mycorollary}	{Corollary}
\newtheorem{myexample}		{Example}
\newtheorem{myconjecture}	{Conjecture}
\newtheorem{myremark}			{Remark}
\newtheorem{myinvariant}	{Invariant}

\numberwithin{mytheorem}		{section}
\numberwithin{mydefinition}	{section}
\numberwithin{mylemma}			{section}
\numberwithin{myproposition}{section}
\numberwithin{mycorollary}	{section}
\numberwithin{myexample}		{section}
\numberwithin{myconjecture}	{section}
\numberwithin{myremark}			{section}
\numberwithin{myinvariant}	{section}

\newenvironment{define}[3][]
{\begin{mydefinition}[#2]\label{#3}#1\normalfont}
{\hfill\ensuremath{\lrcorner}\end{mydefinition}}

\newenvironment{ncdefine}[3][]
{\begin{mydefinition}[#2]\label{#3}#1\normalfont}
{\end{mydefinition}}

\newenvironment{sample}[3][]
{\begin{myexample}[#2]\label{#3}#1\normalfont}
{\hfill\ensuremath{\lrcorner}\end{myexample}}

\newenvironment{ncsample}[3][]
{\begin{myexample}[#2]\label{#3}#1\normalfont}
{\end{myexample}}

\newenvironment{corol}[3][]
{\begin{mycorollary}[#2]\label{#3}#1}
{\hfill\ensuremath{\lrcorner}\end{mycorollary}}

\newenvironment{conject}[3][]
{\begin{myconjecture}[#2]\label{#3}#1\normalfont}
{\hfill\ensuremath{\lrcorner}\end{myconjecture}}

\newenvironment{propose}[3][]
{\begin{myproposition}[#2]\label{#3}#1}
{\hfill\ensuremath{\lrcorner}\end{myproposition}}

\newenvironment{propose2}[2][]
{\begin{myproposition}\label{#2}#1}
{\hfill\ensuremath{\lrcorner}\end{myproposition}}

\newenvironment{ncpropose}[3][]
{\begin{myproposition}[#2]\label{#3}#1}
{\end{myproposition}}

\newenvironment{ncpropose2}[2][]
{\begin{myproposition}\label{#2}#1}
{\end{myproposition}}

\newenvironment{lem}[3][]
{\begin{mylemma}[#2]\label{#3}#1}
{\hfill\ensuremath{\lrcorner}\end{mylemma}}

\newenvironment{thm}[3][]
{\begin{mytheorem}[#2]\label{#3}#1}
{\hfill\ensuremath{\lrcorner}\end{mytheorem}}

\newenvironment{ncthm}[3][]
{\begin{mytheorem}[#2]\label{#3}#1}
{\end{mytheorem}}

\newenvironment{invariant}[3][]
{\begin{myinvariant}[#2]\label{#3}#1\normalfont}
{\hfill\ensuremath{\lrcorner}\end{myinvariant}}

\addtocounter{mytheorem}{0}

%%%%%%%%%%%%%%%%%%%%%%%%%%%%%%%%%%%%%%%%%%%%%%%%%%%%%%%%%%%%%%%%%%%%%%%%%%%%%%%%

% TODO: CONFIGURE core.config.tex USING CONSTANTS IN core.constants.tex

% Import extra packages
\usepackage{subfig}
\usepackage{txfonts} % times
\usepackage{lipsum}
\usepackage{adjustbox}

%%%%%%%%%%%%%%%%%%%%%%%%%%%%%%%%%%%%%%%%%%%%%%%%%%%%%%%%%%%%%%%%%%%%%%%%%%%%%%%%
\begin{document}
%%%%%%%%%%%%%%%%%%%%%%%%%%%%%%%%%%%%%%%%%%%%%%%%%%%%%%%%%%%%%%%%%%%%%%%%%%%%%%%%
%%%%%%%%%%%%%%%%%%%%%%%%%%%%%%%%%%%%%%%%%%%%%%%%%%%%%%%%%%%%%%%%%%%%%%%%%%%%%%%%
% Title
%
% First created on: Jan 5, 2023
% Last updated on:  Jan 5, 2023

% Artem Polyvyanyy
% Email:    artem.polyvyanyy@gmail.com 
% Homepage: http://polyvyanyy.com
%%%%%%%%%%%%%%%%%%%%%%%%%%%%%%%%%%%%%%%%%%%%%%%%%%%%%%%%%%%%%%%%%%%%%%%%%%%%%%%%

% Default commands
\newcommand{\paragraphstart}[2]{{#1}{#2}}

% ARTICLE
\ifthenelse{\equal{\mode}{article}}{%
\date{\today}
\title{\articletitle}
\author{\articleauthor}
\maketitle
\begin{abstract}
Adapted from Springer: An effective title should convey the main topics of the study, highlight the importance of the research, be concise, and attract readers.
To compose a good title, list all the topics/contributions of the paper using as few words as possible and include all the most relevant ones in the title.
A long title will appear clumsy and annoy readers.

\smallskip
\noindent
\textbf{Keywords:} \articlekeyword
\end{abstract}
}%
{%
% LNCS, LNBIP
\ifthenelse{\equal{\mode}{lncs}}{%
\date{\today}
\title{\articletitle}
\titlerunning{\articletitlerunning}
\author{\articleauthor}
\authorrunning{\articleauthorrunning}
\institute{\articleaffiliation}
\maketitle
\begin{abstract}
Adapted from Springer: An effective title should convey the main topics of the study, highlight the importance of the research, be concise, and attract readers.
To compose a good title, list all the topics/contributions of the paper using as few words as possible and include all the most relevant ones in the title.
A long title will appear clumsy and annoy readers.

\smallskip
\noindent
\textbf{Keywords:} \articlekeyword
\end{abstract}
}%
{%
% ICPM
\ifthenelse{\equal{\mode}{icpm}}{%
\date{\today}
\title{\articletitle}
\author{\articleauthor}
\maketitle
\begin{abstract}
Adapted from Springer: An effective title should convey the main topics of the study, highlight the importance of the research, be concise, and attract readers.
To compose a good title, list all the topics/contributions of the paper using as few words as possible and include all the most relevant ones in the title.
A long title will appear clumsy and annoy readers.

\smallskip
\noindent
\textbf{Keywords:} \articlekeyword
\end{abstract}
}%
{%
\ifthenelse{\equal{\mode}{tkde}}{%
\renewcommand{\paragraphstart}[2]{\IEEEPARstart{#1}{#2}}
\title{\articletitle}
\date{\today}
\author{\articleauthor}
\IEEEtitleabstractindextext{%
\begin{abstract}
Adapted from Springer: An effective title should convey the main topics of the study, highlight the importance of the research, be concise, and attract readers.
To compose a good title, list all the topics/contributions of the paper using as few words as possible and include all the most relevant ones in the title.
A long title will appear clumsy and annoy readers.
\end{abstract}
\begin{IEEEkeywords} % Note that keywords are not normally used for peerreview papers.
\articlekeyword
\end{IEEEkeywords}}
\maketitle
}
{
\ifthenelse{\equal{\mode}{aaai}}{%
\title{\articletitle}
\date{\today}
\author{\articleauthor}
\begin{abstract}
Adapted from Springer: An effective title should convey the main topics of the study, highlight the importance of the research, be concise, and attract readers.
To compose a good title, list all the topics/contributions of the paper using as few words as possible and include all the most relevant ones in the title.
A long title will appear clumsy and annoy readers.
\smallskip
\textbf{Keywords:} \articlekeyword
\end{abstract}
\maketitle
}{}}}}}





%%%%%%%%%%%%%%%%%%%%%%%%%%%%%%%%%%%%%%%%%%%%%%%%%%%%%%%%%%%%%%%%%%%%%%%%%%%%%%%%
%%%%%%%%%%%%%%%%%%%%%%%%%%%%%%%%%%%%%%%%%%%%%%%%%%%%%%%%%%%%%%%%%%%%%%%%%%%%%%%%

% Example use of todo commands
\todo{Artem}{Briefly discuss the research area of process mining.}
\chat{Artem}{To accomplish this task, read~\cite{PolyvyanyyPQM2022}.}
\chat{Author}{Thanks, I will do that!}
\done{Author}{Briefly discuss the research area of process mining.}

Use the {$\backslash$change\{old text\}\{new text\}} command to track changes in the manuscript like this: \change{old text}{new text}; where ``old text'' is the old text and ``new text'' is the new text.
Use one of the three versions of the command in \texttt{core.config.tex} to suit your purpose.

Use the {$\backslash$blind\{Artem Polyvyanyy\}} command to remove information that would identify the authors of this manuscript when preparing it for a double-blind review like this: \blind{Artem Polyvyanyy}.
Use {$\backslash$blind[Unblinded text]\{Blinded text\}} to seamlessly switch between unblinded and blinded text like this:
\blind[In our previous work~\cite{PolyvyanyyPQM2022}, we showed that \ldots]{Previously, \citet{PolyvyanyyPQM2022} showed that \ldots}.


%%%%%%%%%%%%%%%%%%%%%%%%%%%%%%%%%%%%%%%%%%%%%%%%%%%%%%%%%%%%%%%%%%%%%%%%%%%%%%%%
\section{Introduction}
\label{sec:introduction}
%%%%%%%%%%%%%%%%%%%%%%%%%%%%%%%%%%%%%%%%%%%%%%%%%%%%%%%%%%%%%%%%%%%%%%%%%%%%%%%%

Write an introduction following the ``funnel'' principle; that is, start with general ideas and gradually specify those into concrete contributions. In the first paragraph, introduce the research area. Use this first paragraph to capture the attention of a \emph{broad audience}. If the reader can not relate to the research area, they will most likely stop reading the paper here. In the second paragraph, introduce the \emph{problem}/\emph{research question(s)} tackled in the paper. Then, list the \emph{requirements} ($R1$, $R2$, and $R3$) for the solution to the problem proposed in this paper and discuss why they are reasonable; unrealistic requirements will diminish the value of the contribution. In the third paragraph, clearly state the \emph{contributions} as you, the author(s), understand them; a bullet point list can be an effective way to present the contributions. Emphasize the \emph{novelty} and \emph{usefulness} of the contributions. Finally, in the last paragraph, introduce the structure of the paper by unveiling the content of the subsequent sections.

%%%%%%%%%%%%%%%%%%%%%%%%%%%%%%%%%%%%%%%%%%%%%%%%%%%%%%%%%%%%%%%%%%%%%%%%%%%%%%%%
\section{Background}
\label{sec:background}
%%%%%%%%%%%%%%%%%%%%%%%%%%%%%%%%%%%%%%%%%%%%%%%%%%%%%%%%%%%%%%%%%%%%%%%%%%%%%%%%

This section should discuss approaches that tackle the same \emph{problem} as your approach but do not address any of the \emph{requirements} ($R1$, $R2$, and $R3$) discussed in the Introduction section.

There is no need to introduce an explicit section entitled ``Background.''
This discussion on the background of your work can be incorporated into other sections, for example, the Introduction and/or Related Work sections.

%%%%%%%%%%%%%%%%%%%%%%%%%%%%%%%%%%%%%%%%%%%%%%%%%%%%%%%%%%%%%%%%%%%%%%%%%%%%%%%%
\section{Related Work}
\label{sec:related:work}
%%%%%%%%%%%%%%%%%%%%%%%%%%%%%%%%%%%%%%%%%%%%%%%%%%%%%%%%%%%%%%%%%%%%%%%%%%%%%%%%

In this section, discuss every existing technique that solves the same \emph{problem} as your approach and addresses a subset of the \emph{requirements} ($R1$, $R2$, and $R3$).
For example, you can write: ``The approach by \citet{PolyvyanyyPQM2022} addresses the problem partially; that is, $R1$ and $R2$, but not $R3$.''

You should discuss all important existing solutions to the \emph{problem} your approach solves.
It is ideal if, through the discussions of this section, you identify existing solutions with which you should compare your approach.

%%%%%%%%%%%%%%%%%%%%%%%%%%%%%%%%%%%%%%%%%%%%%%%%%%%%%%%%%%%%%%%%%%%%%%%%%%%%%%%%
\section{Motivating Example}
\label{sec:motivating:example}
%%%%%%%%%%%%%%%%%%%%%%%%%%%%%%%%%%%%%%%%%%%%%%%%%%%%%%%%%%%%%%%%%%%%%%%%%%%%%%%%

%%%%%%%%%%%%%%%%%%%%%%%%%%%%%%%%%%%%%%%%%%%%%%%%%%%%%%%%%%%%%%%%%%%%%%%%%%%%%%%%
\section{Preliminaries}
\label{sec:preliminaries}
%%%%%%%%%%%%%%%%%%%%%%%%%%%%%%%%%%%%%%%%%%%%%%%%%%%%%%%%%%%%%%%%%%%%%%%%%%%%%%%%

%%%%%%%%%%%%%%%%%%%%%%%%%%%%%%%%%%%%%%%%%%%%%%%%%%%%%%%%%%%%%%%%%%%%%%%%%%%%%%%%
\section{Approach}
\label{sec:approach}
%%%%%%%%%%%%%%%%%%%%%%%%%%%%%%%%%%%%%%%%%%%%%%%%%%%%%%%%%%%%%%%%%%%%%%%%%%%%%%%%

%%%%%%%%%%%%%%%%%%%%%%%%%%%%%%%%%%%%%%%%%%%%%%%%%%%%%%%%%%%%%%%%%%%%%%%%%%%%%%%%
\section{Evaluation}
\label{sec:evaluation}
%%%%%%%%%%%%%%%%%%%%%%%%%%%%%%%%%%%%%%%%%%%%%%%%%%%%%%%%%%%%%%%%%%%%%%%%%%%%%%%%

%%%%%%%%%%%%%%%%%%%%%%%%%%%%%%%%%%%%%%%%%%%%%%%%%%%%%%%%%%%%%%%%%%%%%%%%%%%%%%%%
\section{Conclusion}
\label{sec:conclusion}
%%%%%%%%%%%%%%%%%%%%%%%%%%%%%%%%%%%%%%%%%%%%%%%%%%%%%%%%%%%%%%%%%%%%%%%%%%%%%%%%

%%%%%%%%%%%%%%%%%%%%%%%%%%%%%%%%%%%%%%%%%%%%%%%%%%%%%%%%%%%%%%%%%%%%%%%%%%%%%%%%
\bibliography{bibliography}
%%%%%%%%%%%%%%%%%%%%%%%%%%%%%%%%%%%%%%%%%%%%%%%%%%%%%%%%%%%%%%%%%%%%%%%%%%%%%%%%

%%%%%%%%%%%%%%%%%%%%%%%%%%%%%%%%%%%%%%%%%%%%%%%%%%%%%%%%%%%%%%%%%%%%%%%%%%%%%%%%
\appendix
%%%%%%%%%%%%%%%%%%%%%%%%%%%%%%%%%%%%%%%%%%%%%%%%%%%%%%%%%%%%%%%%%%%%%%%%%%%%%%%%
% Reusabe content on automata
%
% First created on: Nov 13, 2021
% Last updated on:  Nov 13, 2021

% Prepared by Artem Polyvyanyy
% Email:    artem.polyvyanyy@gmail.com 
% Homepage: http://polyvyanyy.com
%%%%%%%%%%%%%%%%%%%%%%%%%%%%%%%%%%%%%%%%%%%%%%%%%%%%%%%%%%%%%%%%%%%%%%%%%%%%%%%%

% core.reuse.automata
% Classical definition of DFA
\begin{ncdefine}{Deterministic Finite Automata}{def:DFAs}{\quad\\}
A \emph{deterministic finite automaton} (DFA) is a tuple $\tuple{Q,\actions,\delta,q_0,A}$, where:
\begin{compactitem}
\item
$Q$ is a finite set of \emph{states};
\item
$\actions$ is a finite set of actions, called the \emph{alphabet};
\item
$\func{\delta}{Q \times \actions}{Q}$ is the \emph{transition function};
\item
$q_0 \in Q$ is the \emph{start state}; and
\item
$A \subseteq Q$ is the set of \emph{accept states}.
\hfill\ensuremath{\lrcorner}
\end{compactitem}
\end{ncdefine}

%%%%%%%%%%%%%%%%%%%%%%%%%%%%%%%%%%%%%%%%%%%%%%%%%%%%%%%%%%%%%%%%%%%%%%%%%%%%%%%%


%%%%%%%%%%%%%%%%%%%%%%%%%%%%%%%%%%%%%%%%%%%%%%%%%%%%%%%%%%%%%%%%%%%%%%%%%%%%%%%%
\section{Petri Nets (Reuse)}
\label{sec:reuse:petri:nets}
%%%%%%%%%%%%%%%%%%%%%%%%%%%%%%%%%%%%%%%%%%%%%%%%%%%%%%%%%%%%%%%%%%%%%%%%%%%%%%%%

A Petri net is a model of a distributed system.

\begin{define}{Petri net}{def:petri:net}{\quad\\}
A \emph{(labeled) Petri net}, or a \emph{net}, $N$ is a quintuple $\tuple{P,T,F,\Lambda,\lambda}$, where
$P$ is a finite set of \emph{places},
$T$ is a finite set of \emph{transitions},
$F \subseteq (P \times T) \cup (T \times P)$ is the \emph{flow relation},
$\Lambda$ is a set of \emph{labels}, such that $\tau \in \Lambda$ is the \emph{silent label} and sets $P$, $T$, and $\Lambda$ are pairwise disjoint, and 
$\func{\lambda}{T}{\Lambda}$ is the \emph{labeling function}.
\end{define}

\noindent
If $\funcCall{\lambda}{t}=\tau$, $t \in T$, then $t$ is \emph{silent}; otherwise $t$ is \emph{observable}.
Observable transitions represent activities from the problem domain, whereas silent transitions encode internal actions of the system. 
A \emph{marking} of a net encodes its state.

\begin{define}{Marking}{def:marking}{\quad\\}
A \emph{marking} $M$ of a net $\tuple{P,T,F,\Lambda,\lambda}$ is a multiset over places $P$.
\end{define}

\noindent
A net system is a net with an initial and final markings.

\begin{define}{Net system}{def:net:system}{\quad\\}
A \emph{net system}, or a \emph{system}, $S$ is a triple $\tuple{N,M_{\mathit{ini}},M_{\mathit{fin}}}$, where 
$N=\tuple{P,T,F,\Lambda,\lambda}$ is a net,
$M_{\mathit{ini}}$ is the \emph{initial marking} of $N$, and 
$M_{\mathit{fin}}$ is the \emph{final marking} of $N$.
\end{define}

\noindent
Let $N=\tuple{P,T,F,\Lambda,\lambda}$ be a net.
A transition $t \in T$ is \emph{enabled} in a marking $M$ of $N$, denoted by $\enabled{N}{M}{t}$, if every \emph{input place} of $t$ contains at least one token, \ie 
$\predicate{\forall}{p \in \setbuilder{x \in P}{(x,t) \in F}}{\funcCall{M}{p} > 0}$. 
An enabled transition $t \in T$ can \emph{occur}.
An occurrence of $t$ \emph{leads to} a fresh marking $M' = (M \setminus \setbuilder{x \in P}{(x,t) \in F}) \uplus \setbuilder{y \in P}{(t,y) \in F}$ of $N$, denoted by $\occurrence{N}{M}{t}{M'}$.

A finite sequence of transitions $\sigma=\sequence{t_1,t_2,\ldots,t_n} \in \kleenestar{T}$, $n \in \natnumwithzero$, is an \emph{occurrence sequence} of a net $N$ at marking $M$, if $\sigma$ is empty or there exists a sequence of markings $\sequence{M_0,M_1,\ldots,M_n}$, such that $M_0=M$ and for every position $i \in \intintervalcc{1}{n}$ in $\sigma$ it holds that $\occurrence{N}{M_{i-1}}{t_i}{M_i}$.
We say that $\sigma$ \emph{leads $N$ from $M$ to $M_n$} and denote this fact by $\leadsfromto{N}{M}{\sigma}{M_n}$.
Finally, $\sigma$ is an \emph{execution} of a net system $\tuple{N,M_\mathit{ini},M_\mathit{fin}}$ if $\sigma$ leads $N$ from $M_\mathit{ini}$ to $M_\mathit{fin}$.
%%%%%%%%%%%%%%%%%%%%%%%%%%%%%%%%%%%%%%%%%%%%%%%%%%%%%%%%%%%%%%%%%%%%%%%%%%%%%%%%
\newpage
\listoftodos
%%%%%%%%%%%%%%%%%%%%%%%%%%%%%%%%%%%%%%%%%%%%%%%%%%%%%%%%%%%%%%%%%%%%%%%%%%%%%%%%
\end{document}
%%%%%%%%%%%%%%%%%%%%%%%%%%%%%%%%%%%%%%%%%%%%%%%%%%%%%%%%%%%%%%%%%%%%%%%%%%%%%%%%