%%%%%%%%%%%%%%%%%%%%%%%%%%%%%%%%%%%%%%%%%%%%%%%%%%%%%%%%%%%%%%%%%%%%%%%%%%%%%%%%
% Improting useful packages and defining useful commands for LaTeX
%
% First created on: Dec 2, 2019
% Last updated on:  Feb 4, 2025
%%%%%%%%%%%%%%%%%%%%%%%%%%%%%%%%%%%%%%%%%%%%%%%%%%%%%%%%%%%%%%%%%%%%%%%%%%%%%%%%

%%%%%%%%%%%%%%%%%%%%%%%%%%%%%%%%%%%%%%%%%%%%%%%%%%%%%%%%%%%%%%%%%%%%%%%%%%%%%%%%
% USE PACKAGES
%%%%%%%%%%%%%%%%%%%%%%%%%%%%%%%%%%%%%%%%%%%%%%%%%%%%%%%%%%%%%%%%%%%%%%%%%%%%%%%%

%-------------------------------------------------------------------------------
% xparse – A generic document command parser
% https://ctan.org/pkg/xparse?lang=en
\usepackage{xparse}
%-------------------------------------------------------------------------------

%-------------------------------------------------------------------------------
% ulem – Package for underlining
% https://ctan.org/pkg/ulem?lang=en
\usepackage[normalem]{ulem}
%-------------------------------------------------------------------------------

% Use natbib package
\usepackage[square,sort&compress,comma,numbers]{natbib}
\renewcommand{\bibfont}{\small}
\renewcommand{\bibsep}{0pt}
\renewcommand{\bibsection}{\section*{References}\vspace{-2mm}}

%-------------------------------------------------------------------------------
%\usepackage{amsthm}
%\usepackage{amssymb}
%-------------------------------------------------------------------------------
% A comprehensive (SI) units package
% https://ctan.org/pkg/siunitx?lang=en
%\usepackage{siunitx}
%\sisetup{
%group-separator={,},
%round-mode=places,
%round-precision=3,
%table-format=1.3,
%}
%-------------------------------------------------------------------------------
% Useful table packages
% booktabs - the package enhances the quality of tables
% https://ctan.org/pkg/booktabs?lang=en
% multirow - create tabular cells spanning multiple rows
% https://ctan.org/pkg/multirow?lang=en
\usepackage{booktabs}
\usepackage{multirow}
%-------------------------------------------------------------------------------
% The package typesets fractions “nicely” — in the form ‘a/b’
% https://ctan.org/pkg/nicefrac?lang=en
% USAGE: $\nicefrac{1}{2}$
\usepackage{nicefrac}
%-------------------------------------------------------------------------------
% https://ctan.org/pkg/ifthen?lang=en
\usepackage{ifthen}
%-------------------------------------------------------------------------------
% https://en.wikibooks.org/wiki/LaTeX/Colors
\usepackage[usenames,dvipsnames]{xcolor}
%-------------------------------------------------------------------------------
%https://ctan.org/pkg/algorithm2e?lang=en
\usepackage[algoruled,vlined,linesnumbered]{algorithm2e}
%-------------------------------------------------------------------------------
% https://ctan.org/pkg/amsmath?lang=en
%
% Recommended as an adjunct to serious mathematical typesetting in LATEX.
\usepackage{amsmath}
%-------------------------------------------------------------------------------
% https://ctan.org/pkg/amsfonts?lang=en
%
% An extended set of fonts for use in mathematics, including: extra mathematical 
% symbols; blackboard bold letters (uppercase only); fraktur letters; subscript 
% sizes of bold math italic and bold Greek letters; subscript sizes of large 
% symbols such as sum and product; added sizes of the Computer Modern small caps
% font; cyrillic fonts (from the University of Washington); Euler mathematical 
% fonts.
\usepackage{amsfonts}
%-------------------------------------------------------------------------------
\usepackage{mdframed}
%-------------------------------------------------------------------------------
% https://ctan.org/pkg/paralist?lang=en
\usepackage{paralist}
\setdefaultleftmargin{1em}{1em}{1em}{1em}{1em}{1em}
%-------------------------------------------------------------------------------
\usepackage{datetime}
\newdateformat{monthyeardate}{\monthname[\THEMONTH] \THEYEAR}
%-------------------------------------------------------------------------------
\usepackage[algoruled]{algorithm2e}
%-------------------------------------------------------------------------------
% Manage to do items
% https://ctan.org/pkg/todonotes?lang=en
\ifshowtodos
\usepackage[textsize=scriptsize,colorinlistoftodos]{todonotes}
\else
\usepackage[textsize=scriptsize,colorinlistoftodos,disable]{todonotes} % disable all todo notes
\fi
% useful commands
% \missingfigure{My comment} - specify a missing figure
% \listoftodos - list all todos in the current document
\let\todonote\todo
\renewcommand{\todo}[2]{\todonote[inline,color=red!20]{TODO (#1): #2}}
\newcommand{\done}[2]{\todonote[inline,color=green!20]{DONE (#1): #2}}
\newcommand{\chat}[2]{\todonote[inline,color=black!20,nolist]{#1: #2}}
\newcommand{\idea}[2]{\todonote[inline,color=blue!20,nolist]{IDEA (#1): #2}}
%-------------------------------------------------------------------------------
\usepackage{tikz,pgf,xcolor}
\usetikzlibrary{arrows,automata,trees,plotmarks,shadows,shapes}
\usepackage{pgfplots}
\usetikzlibrary{pgfplots.groupplots}
%-------------------------------------------------------------------------------
% TODO: check the usage
%\usepackage{url}
%\urlstyle{same}
%-------------------------------------------------------------------------------
% TODO: check the usage
\usepackage{mathtools} 
%-------------------------------------------------------------------------------
% Manage references
% https://ctan.org/tex-archive/macros/latex/contrib/cleveref
\usepackage[capitalise,nameinlink]{cleveref}
%-------------------------------------------------------------------------------
% Flow text around small figures
\usepackage{wrapfig}
%-------------------------------------------------------------------------------

% Can be useful
%http://anorien.csc.warwick.ac.uk/mirrors/CTAN/macros/latex/contrib/constants/constants.pdf

%%%%%%%%%%%%%%%%%%%%%%%%%%%%%%%%%%%%%%%%%%%%%%%%%%%%%%%%%%%%%%%%%%%%%%%%%%%%%%%%
% CONSTANTS
%%%%%%%%%%%%%%%%%%%%%%%%%%%%%%%%%%%%%%%%%%%%%%%%%%%%%%%%%%%%%%%%%%%%%%%%%%%%%%%%

\definecolor{mybluecolor}{RGB}{50,106,218}
\definecolor{myredcolor}{RGB}{176,53,53}
\definecolor{mygreencolor}{RGB}{93,172,0}
\definecolor{myyellowcolor}{RGB}{255,163,34}
\definecolor{mypurplecolor}{RGB}{86,35,132}
\definecolor{mytealcolor}{RGB}{30,161,165}

\newcommand{\redtext}[1]{\textcolor{myredcolor}{{#1}}}
\newcommand{\bluetext}[1]{\textcolor{mybluecolor}{{#1}}}

%%%%%%%%%%%%%%%%%%%%%%%%%%%%%%%%%%%%%%%%%%%%%%%%%%%%%%%%%%%%%%%%%%%%%%%%%%%%%%%%
% DEFINE NEW COMMANDS
%%%%%%%%%%%%%%%%%%%%%%%%%%%%%%%%%%%%%%%%%%%%%%%%%%%%%%%%%%%%%%%%%%%%%%%%%%%%%%%%

\newcommand{\splitatcommas}[1]{%
  \begingroup
  \ifnum\mathcode`,="8000
  \else
    \begingroup\lccode`~=`, \lowercase{\endgroup
      \edef~{\mathchar\the\mathcode`, \penalty-100 \noexpand\hspace{-1pt plus 3em}}%
    }\mathcode`,="8000
  \fi
  #1%
  \endgroup
}

%-------------------------------------------------------------------------------
% Arithmetic operations
\newcommand{\mult}{{\ensuremath \,\cdot\,}}

% Math symbols
\providecommand{\bigsum}{\ensuremath \mbox{\LARGE $\mathsurround0pt\sum$}}
\renewcommand	 {\bigsum}{\ensuremath \mbox{\LARGE $\mathsurround0pt\sum$}}

% Math functions
\newcommand{\func}[3]{{{#1}:{#2} \rightarrow {#3}}}
\newcommand{\funcCall}[2]{{\ensuremath {\mathit{#1}}_{\!}\left({#2}\right)}}
\newcommand{\funcDomain}[1]{{\ensuremath {\mathit{dom}}_{\!}\left({#1}\right)}}
\newcommand{\funcRange}[1]{{\ensuremath {\mathit{rng}}_{\!}\left({#1}\right)}}

% Math paired delimiters
\DeclarePairedDelimiter\ceil{\lceil}{\rceil}
\DeclarePairedDelimiter\floor{\lfloor}{\rfloor}

% Math intervals
\newcommand{\intintervalcc}[2]{{\ensuremath \left[#1 \,..\, #2\right]}}
\newcommand{\intervalcc}[2]{{\ensuremath \left[#1, #2\right]}}

% Math powerset
\providecommand{\powerset}[1]{{\ensuremath \mathcal{P}\!\left({#1}\right)}}
\renewcommand{\powerset}[1]{{\ensuremath \mathcal{P}\!\left({#1}\right)}}

% Math multiset
\providecommand{\support}[1]{{\ensuremath \mathit{Supp}\left({#1}\right)}}
\renewcommand{\support}[1]{{\ensuremath \mathit{Supp}\left({#1}\right)}}
\providecommand{\cardinality}[1]{\ensuremath \left|{#1}\right|}
\renewcommand{\cardinality}[1]{\ensuremath \left|{#1}\right|}
\providecommand{\multiplicity}[2]{\ensuremath \funcCall{m_{#1}}{#2}}
\renewcommand{\multiplicity}[2]{\ensuremath \funcCall{m_{#1}}{#2}}
 \newcommand{\mset}[1] {\ensuremath [\splitatcommas{#1}]}
\newcommand{\msetel}[2]{{\ensuremath {{#1}^{#2}}}}

% Math sets
\newcommand{\set}[1]{\ensuremath \left\{\splitatcommas{#1}\right\}}
\newcommand{\setbuilder}[2]{\ensuremath \left\{ #1 \,|\, #2 \right\}}

% Math tuple functions
\newcommand{\pair}[2]{\ensuremath \left(\splitatcommas{#1, #2}\right)}
\newcommand{\triple}[3]{\ensuremath \left(\splitatcommas{#1, #2, #3}\right)}
\newcommand{\tuple}[1]{\ensuremath \left(\splitatcommas{#1}\right)}

% Math sequences
\newcommand{\kleenestar}[1]{{\ensuremath {#1}^{*}}}
\newcommand{\emptysequence}{{\ensuremath \epsilon}}
\newcommand{\sequence}[1]{\ensuremath \langle\splitatcommas{#1}\rangle}
\newcommand{\seqLength}[1]{\ensuremath \left|{#1}\right|}
\newcommand{\concat}[2]{\ensuremath #1 \circ #2}
\newcommand{\subseq}[3]{\ensuremath \funcCall{subseq}{#1, #2, #3}}
\newcommand{\subseqid}[3]{\ensuremath \triple{#1}{#2}{#3}}
\newcommand{\crossover}[3]{\ensuremath #1 \,\otimes\, #2 \,=\, #3}
\newcommand{\crossoverop}[2]{\ensuremath #1 \otimes #2}

% Petri nets
\newcommand{\enabled}[3]{\ensuremath \pair{#1}{#2}\left[#3\right\rangle}
\newcommand{\occurrence}[4]{\ensuremath \pair{#1}{#2}\left[#3\right\rangle\pair{#1}{#4}}
\newcommand{\leadsfromto}[4]{\ensuremath \pair{#1}{#2}\left[#3\right\rangle\pair{#1}{#4}}

% Predicates
\newcommand{\predicate}[3]{\ensuremath #1 \, #2 : #3}

% Theorems and proofs
\providecommand{\qed}{\hfill\ensuremath{\filledsquare}}
\renewcommand  {\qed}{\hfill\ensuremath{\filledsquare}}
\providecommand{\holds}{\ensuremath :}
\renewcommand	 {\holds}{\ensuremath :}
\providecommand{\iff}{\ensuremath \Leftrightarrow}
\renewcommand	 {\iff}{\ensuremath \Leftrightarrow}
\providecommand{\implies}{\ensuremath \Rightarrow}
\renewcommand	 {\implies}{\ensuremath \Rightarrow}
\providecommand{\bigforall}{\ensuremath \mbox{\LARGE $\mathsurround0pt\forall$}}
\renewcommand	 {\bigforall}{\ensuremath \mbox{\LARGE $\mathsurround0pt\forall$}}

% Logs
\newcommand{\action}[1]{\texttt{#1}}
\newcommand{\trace}[1]{\texttt{#1}}

%-------------------------------------------------------------------------------

% Figure functions
% Use: \fig{pos}{vspace(mm)}{scale}{trim(0 0 0 0)}{figure}{vspace(mm)}{Caption}{vspace(mm)}{fig:label}
\newcommand{\fig}[9]{\begin{figure}[#1]
\vspace{#2mm}
\begin{center}
	\includegraphics[scale=#3,trim=#4]{#5}
\end{center}
\vspace{#6mm}
\caption{#7.}
\vspace{#8mm}
\label{#9}
\end{figure}}

%-------------------------------------------------------------------------------

% An initial attempt to fix orphan lines in a pargraph
\newcommand\fixit[2][.05]{%
  \setbox0=\hbox{\hspace{\parindent}#2}\fixithelp{#1}{#2}}
\newcommand\fixithelp[2]{%
  \wd0=\dimexpr\wd0-\linewidth\relax%
  \ifdim\wd0>0pt\relax%
    \fixithelp{#1}{#2}%
  \else%
    \wd0=\dimexpr\wd0+\linewidth\relax
    \ifdim\wd0>.9\linewidth\relax%
      {\parfillskip0pt\relax#2\par}%
    \else%
      \ifdim\wd0>.8\linewidth\relax%
        {\parfillskip0pt\relax#2\hspace{.2\linewidth}\par}%
      \else%
        \ifdim\wd0<#1\linewidth\relax%
          {\parfillskip0pt\relax#2\par}%
        \else%
          \ifdim\wd0<.2\linewidth\relax%
            {\parfillskip0pt\relax#2\hspace{.8\linewidth}\mbox{}\par}%
          \else%
            #2%
          \fi
        \fi
      \fi
    \fi
  \fi%
}
