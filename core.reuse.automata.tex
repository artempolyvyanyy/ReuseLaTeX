%%%%%%%%%%%%%%%%%%%%%%%%%%%%%%%%%%%%%%%%%%%%%%%%%%%%%%%%%%%%%%%%%%%%%%%%%%%%%%%%
% Reusabe content on automata
%
% First created on: Nov 13, 2021
% Last updated on:  Nov 13, 2021

% Prepared by Artem Polyvyanyy
% Email:    artem.polyvyanyy@gmail.com 
% Homepage: http://polyvyanyy.com
%%%%%%%%%%%%%%%%%%%%%%%%%%%%%%%%%%%%%%%%%%%%%%%%%%%%%%%%%%%%%%%%%%%%%%%%%%%%%%%%

% core.reuse.automata
% Classical definition of DFA
\begin{ncdefine}{Deterministic Finite Automata}{def:DFAs}{\quad\\}
A \emph{deterministic finite automaton} (DFA) is a tuple $\tuple{Q,\actions,\delta,q_0,A}$, where:
\begin{compactitem}
\item
$Q$ is a finite set of \emph{states};
\item
$\actions$ is a finite set of actions, called the \emph{alphabet};
\item
$\func{\delta}{Q \times \actions}{Q}$ is the \emph{transition function};
\item
$q_0 \in Q$ is the \emph{start state}; and
\item
$A \subseteq Q$ is the set of \emph{accept states}.
\hfill\ensuremath{\lrcorner}
\end{compactitem}
\end{ncdefine}

%%%%%%%%%%%%%%%%%%%%%%%%%%%%%%%%%%%%%%%%%%%%%%%%%%%%%%%%%%%%%%%%%%%%%%%%%%%%%%%%

